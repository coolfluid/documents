\documentclass[12pt]{scrartcl}
\usepackage{geometry}
\geometry{a4paper}


\title{Dynamic Object Oriented Data Structure for Scientific High Performance Computing}
% or how a data structure can be constructed and be adapted at run-time

\author{}
\begin{document}
  \maketitle  
  \tableofcontents
  
  \section{Motivation}
  % Why would we want a dynamic adaptable data structure?
  
  \section{Component based data structure}
  
  \subsection{Browsing of data structure}
  % or how a data structure can be seen visually
  
  % Explain the need for scientific "USERS" to see how to access and understand the available data,
  % as well as developers to not lose track of what is there and what is missing.

  % Other tree-based datastructures: HDF, CGNS


  \subsection{Dynamic Organization}
  % or how the datastructure is organized
  
  % 1) Creation
  % 2) Reorganization
  % 3) Growth/Consolidation
  %    --> mathematical


  
  \subsection{Navigation and Query of data structure}
  % or how to access the data structure
  
  % 1) path access
  % 2) member functions
  % 3) find functions
  % 4) looping by type, name, tag, or other given predicate  (transparency, linearization)
  



  \section{A Mesh data structure in coolfluid 3}
  % or how a Mesh in cf3 is defined
  
      \subsection{Requirements}
      % or what is needed in terms of cf3 to be able to use any type of numerics with the mesh data structure
      
      % 1) Flexible, adaptible, simple --> mesh-transformations
      % 2) High-order numerics need high-order representations of mesh
      % 3) Discontinuous and Continuous fields
      % 4) High performance computing 
      %    --> minimal cache missing
      %    --> parallel distrubuted partitioning, overlap growing, renumbering
      % 5) Algorithm optimization (algorithms per element type)
      % 6) Multiple meshes (interpolation, probe, multigrid)
      
      
      \subsection{Organization}
      % or how is the mesh organized and 
        
        % 1) CAD-like topology,
        %    or how a mesh can be presented in a nested fashion
        
        % 2) Subdivision of elements per type
        %    - Optimized algorithms (e.g. interpolation, template based)
        %    - Connectivity tables are completely filled
        
        % 3) Dictionaries to consolidate and inform nodes and fields
        
        % 4) Dynamic growth, adaptation (see generic section of Organization)
      
      \subsection{Fields}
      % or how spaces and dictionaries work together to manage e.g. high-order fields 
      
        % 1) What is a Space --> View of same mesh adapted to specifics of given numerics
        % 2) What defines a space
        % 3) What a space allows to do
        %      - interpolation
        %      - high-order fields, low-order mesh --> decoupling
        %      - discontinuous fields, continuous mesh --> again decoupling
        %      - reuse of mesh topology --> filtering on region
      
    
      \subsection{Capabilities and examples/applications}
      % or what can we do with the mesh in cf3
  
\end{document}