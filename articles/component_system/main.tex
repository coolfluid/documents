\documentclass[12pt]{scrartcl}

\usepackage{geometry} 
\geometry{a4paper} 

%%%-----------------------------------------------------------------------

\title{A Component System for High Performance Computing}

\author{}

%%%-----------------------------------------------------------------------

\begin{document}

\maketitle

\tableofcontents

%%%-----------------------------------------------------------------------

\section{Motivation} 

% or what problems we are about to solve

%%%-----------------------------------------------------------------------

\section{Dynamic Components} 

% how we make dynamic components on a static language ?
% why we need still a static library, and dont use directly python/numpy/... ?

\subsection{Type mechanism}

% why we need a type mechanism - platform independent ?

% factories and libraries

% builders 

% inheritance type

\subsection{Signals}


\subsection{Reflection and Introspection}

\subsection{Serialization} % and the xml protocol

%%%-----------------------------------------------------------------------

\section{Identity and Structure} 

% how we organize the components between each other ?
% how we identify the components ?
% how we address and transverse the components?

\subsection{Tree structure} 

% and the analogy to a file system

% multiple roots & the question of free standing components

\subsection{Addressing}

% cpath

% compoentn access

% links

% uuid's

\subsection{Iterators}

% tree transversal

% finding + type  mechanism


%%%-----------------------------------------------------------------------

\section{Interfaces to the Component System}

% or how we can interface this system with other languages and UI's

%%%-----------------------------------------------------------------------

\section{Active Objects} 

% can componets or sets of components become active objects?

%%%-----------------------------------------------------------------------

% \section{SIMULYNK}

% can we really make this a modular simultation???
% what would the connections represent? 
% messages, signals, data flow?

%%%-----------------------------------------------------------------------


\end{document}