\documentclass[12pt]{scrartcl}

\usepackage{geometry} 
\geometry{a4paper} 

% Engineering with Computers - Springer
% Computing in Science and Engineering - AIP

%%%-----------------------------------------------------------------------

\title{A Dynamic Component System for High Performance Computing}

\author{}

%%%-----------------------------------------------------------------------

\begin{document}

\maketitle

\tableofcontents

%%%-----------------------------------------------------------------------

\section{Motivation} 

% or what problems we are about to solve

%%%-----------------------------------------------------------------------

\section{Dynamic Components} 

% how we make dynamic components on a static language ?
% why we need still a static library, and dont use directly python/numpy/... ?

\subsection{Type mechanism}

% why we need a type mechanism - platform independent ?

% factories and libraries

% builders 

% inheritance type

\subsection{Signals}

% provide dynamic interfaces that can mutate at run-time
% as messages in objective-c

% anonymous signals are Events centralized in an Event manager

% dichotomy signals vs Actions (diff granularity)


\subsection{Properties and Options}

% properties as dynamic variables

% options & configuration system


\subsection{Reflection and Introspection}

% signals and properties are registered within component

% obtain signal signatures via a signal itself - enables introspection  



\subsection{Communication Protocol} 

% serialization

% json, xml, google protocol buffers


%%%-----------------------------------------------------------------------

\section{Component Identity and Structure} 

% how we organize the components between each other ?
% how we identify the components ?
% how we address and transverse the components?

\subsection{Tree structure} 

% and the analogy to a file system

% multiple roots & the question of free standing components

% data managemets - data as components

\subsection{Ownership}

% strong ownership of children components in tree

% plus access to other components via handles

\subsection{Addressing}

% cpath

% compoentn access

% links

% uuid's

\subsection{Iterators}

% tree transversal

% finding + type  mechanism


%%%-----------------------------------------------------------------------

\section{Interfaces to the Component System}

% how we can interface this system with other languages and UI's ?

\subsection{Dynamic languages}

% python has a shallow / simple interface

\subsection{Component Shell}

% login into a component and interaact directly with him

\subsection{Graphical Interface}

% automatic generation of the interface

%%%-----------------------------------------------------------------------

\section{Applications}

%%%-----------------------------------------------------------------------

%%% TO COMPLEMENT 

% distributed shared memory model - parallel environment

% solver design

% network layer - active objects

%%%-----------------------------------------------------------------------

%%% TODO:

% active objects + network layer + shell login to comp.
% multpi-physics

\end{document}